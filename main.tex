\documentclass[a4paper,12pt]{scrartcl}
\usepackage{klausur}

% Parameter setzen
\setklasse{5a}
\settitel{1. Schulaufgabe im Fach Mathematik}
\setdatum{08. Dezember 2024}
\setarbeitszeit{90}

\begin{document}
\klausurtitel

\begin{aufgabe}{Berechnen Sie die Lösungen der folgenden Gleichungen.}
  \teilaufgabe[2]{Berechnen Sie $x$ für die Gleichung $x^2 - 4x + 4 = 0$.}
  \teilaufgabe[2]{Berechnen Sie $x$ für die Gleichung $x^2 - 4x + 4 = 0$.}
  \aufgabentext{
  \begin{myplot}{-5}{-1.5}{4.5}{4.1}
    \plot{sin(deg(\x))+1}{G_f}
    \plotvonbis{exp(\x)}{G_g}{-5}{1.4}
    \plotgrid
    \hachse{t}
    \yachse
    \achsenbeschriftung
  \end{myplot}
  }
  \teilaufgabe[3]{Was zeigt das Bild?}
\end{aufgabe}

\begin{aufgabe}{Berechne: \euro}
  \teilaufgabe[2] {$\eur{5.2} + \num{1860989}$}
  \teilaufgabe[9] { $\int_0^1 x dx$
  } 
\end{aufgabe}
\begin{aufgabe}[5]{Eine Aufgabe ohne Teilaufgaben}
  \aufgabentext{Hier steht Ihr Text.}
\end{aufgabe}

\gesamtpunkte
\end{document}
