\documentclass[a4paper,12pt]{scrartcl}
\usepackage{klausur}

% Parameter setzen
\setklasse{5a}
\settitel{1. Schulaufgabe im Fach Mathematik}
\setdatum{08.12.2024}
\setarbeitszeit{90}

\begin{document}
\klausurtitel

\begin{aufgabe}{Berechnen Sie die Lösungen der folgenden Gleichungen. Jetzt mache ich den Text extra lang. Passt das immer noch?}
  \teilaufgabe[2]{Berechnen Sie $x$ für die Gleichung $x^2 - 4x + 4 = 0$.}
  \teilaufgabe[2]{Berechnen Sie $x$ für die Gleichung $x^2 - 4x + 5 = 1$. Tanze danach deinen Namen und singe ein Lied dazu.}
  \aufgabentext{\textbf{Betrachten Sie jetzt das Bildchen}\\
  \begin{myplot}{-5}{-1.5}{4.5}{4.1}
    \plot{sin(deg(\x))+2}{G_f}
    \plotvonbis{exp(\x)}{G_g}{-5}{1.4}
    \plotgrid
    \hachse{t}
    \yachse
    \achsenbeschriftung
  \end{myplot}
  }
  \teilaufgabe[3]{Was zeigt das Bild?}
\end{aufgabe}

\begin{aufgabe}{Berechne: \euro}
  \teilaufgabe[2] {$\eur{5.2} + \num{1860989}$}
  \teilaufgabe[9] { $\int_0^1 x dx$}
  \teilaufgabe[2]{Zeichne ins Koodinatensystem ein.}
  \aufgabentext{
\begin{myplot}{-3}{-2}{10}{5}
  \xachse
  \yachse
  \plotgrid
  \achsenbeschriftung
\end{myplot}
  }
\end{aufgabe}
\noindent\textbf{Bitte wenden!}
\newpage
\begin{aufgabe}[5]{Eine Aufgabe ohne Teilaufgaben. Der Text ist auch wieder sehr lang um das Layout zu sprengen.}
  \aufgabentext{Hier steht Ihr Text.
  \nlgrid{5}
  }
\end{aufgabe}

\begin{aufgabe}[88]{Gehe nach Hause.}
\end{aufgabe}

\gesamtpunkte
\end{document}
